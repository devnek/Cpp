\documentclass{article}
\usepackage{hyperref}
\usepackage{listings}
\usepackage{color}

\definecolor{codebg}{rgb}{0.95,0.95,0.95}
\lstset{
	backgroundcolor=\color{codebg},
	basicstyle=\ttfamily,
	frame=single,
	breaklines=true,
	postbreak=\mbox{\textcolor{red}{$\hookrightarrow$}\space},
}

\title{Introduction to Python and Virtual Environments}
\author{}
\date{}

\begin{document}
	
	\maketitle
	
	\section{Introduction to Python}
	
	Python is a high-level, interpreted programming language known for its simplicity and readability. It supports multiple programming paradigms, including procedural, object-oriented, and functional programming. Python is widely used in various domains such as web development, data analysis, artificial intelligence, scientific computing, and more.
	
	\subsection{Key Features of Python}
	\begin{itemize}
		\item \textbf{Easy to Read and Write}: Python's syntax is designed to be readable and straightforward.
		\item \textbf{Interpreted Language}: Python code is executed line by line, which makes debugging easier.
		\item \textbf{Dynamic Typing}: Variables in Python do not require an explicit declaration to reserve memory space.
		\item \textbf{Extensive Standard Library}: Python comes with a rich standard library that supports many common programming tasks.
		\item \textbf{Community Support}: Python has a large and active community, contributing to a vast ecosystem of libraries and frameworks.
	\end{itemize}
	
	\section{Common Python Libraries by Use Case}
	
	\subsection{Web Development}
	\begin{itemize}
		\item \textbf{Django}: A high-level Python web framework that encourages rapid development.
		\item \textbf{Flask}: A micro web framework for building small to medium-sized web applications.
	\end{itemize}
	
	\subsection{Data Analysis}
	\begin{itemize}
		\item \textbf{Pandas}: Provides data structures and data analysis tools.
		\item \textbf{NumPy}: Supports large, multi-dimensional arrays and matrices.
	\end{itemize}
	
	\subsection{Machine Learning}
	\begin{itemize}
		\item \textbf{Scikit-learn}: A library for simple and efficient tools for data mining and data analysis.
		\item \textbf{TensorFlow} and \textbf{PyTorch}: Libraries for deep learning and neural networks.
	\end{itemize}
	
	\subsection{Scientific Computing}
	\begin{itemize}
		\item \textbf{SciPy}: Used for scientific and technical computing.
		\item \textbf{Matplotlib}: A plotting library for creating static, interactive, and animated visualizations.
	\end{itemize}
	
	\subsection{Web Scraping}
	\begin{itemize}
		\item \textbf{BeautifulSoup}: A library for parsing HTML and XML documents.
		\item \textbf{Scrapy}: A framework for extracting data from websites.
	\end{itemize}
	
	\subsection{Automation}
	\begin{itemize}
		\item \textbf{Selenium}: A tool for automating web browsers.
		\item \textbf{PyAutoGUI}: A library for programmatically controlling the mouse and keyboard.
	\end{itemize}
	
	\section{Virtual Environments}
	
	A virtual environment in Python is a self-contained directory that contains a Python installation for a particular version along with several additional packages. Virtual environments are useful for:
	
	\begin{itemize}
		\item \textbf{Dependency Management}: Ensures that each project has its own dependencies, avoiding conflicts between projects.
		\item \textbf{Version Control}: Allows different projects to use different versions of the same library, which is particularly useful when working on multiple projects simultaneously.
		\item \textbf{Isolation}: Prevents system-wide installations of Python packages, reducing the risk of affecting other projects.
		\item \textbf{Reproducibility}: Makes it easier to share projects with others, as the environment can be replicated using a \texttt{requirements.txt} file.
	\end{itemize}
	
	\subsection{Creating and Using a Virtual Environment}
	
	To create a virtual environment, you can use the \texttt{venv} module, which is included in Python 3.3 and later:
	
	\begin{lstlisting}[language=bash]
		# Create a virtual environment named 'env'
		python -m venv env
		
		# Activate the virtual environment
		# On Windows
		.\env\Scripts\activate
		# On macOS/Linux
		source env/bin/activate
		
		# Deactivate the virtual environment
		deactivate
	\end{lstlisting}
	
	By using virtual environments, you maintain a clean and efficient workspace for your Python projects, ensuring that each project has the necessary dependencies without interference from others.
	
\end{document}